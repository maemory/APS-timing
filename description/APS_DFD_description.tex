\documentclass[12pt]{article}

\usepackage{amsmath}
\usepackage{amssymb}
\usepackage{amsfonts}
\usepackage{epsfig}

\renewcommand{\theequation}{\thesection.\arabic{equation}}
\newcommand{\newsection}{\setcounter{equation}{0}\section}

%%\renewcommand{\baselinestretch}{2.0} % for double spacing use {2}

\textheight 9.0truein
\parskip 0in
\topmargin -0.5truein \textwidth 6.5truein \oddsidemargin -0.05in
\evensidemargin -0.05in
%\renewcommand{\baselinestretch}{2}   %line space adjusted here
\setcounter{footnote}{0}


\begin{document}

\begin{center}
{\large \bf APS DFD Timing Code}
\\[0.05cm]
\end{center}

%\noindent \textbf{{\large 1 General Description}}
%    \vskip0.01cm

\noindent  In general, the APS\_DFD is a timer for conference presentation.  It comes with several matlab programs that run a excel spreedsheet that contains the schedule of the entire conference.  The program generates a screen that display the real-time schedule of the conference.  The following is a to-do list to run the program:

\begin{itemize}
 \item[1.]  Make sure the following list of programs are in your current directory:
 \begin{itemize}
  \item APS\_DFD.m
  \item session.m
  \item loadTimingData.m
  \item update\_delay.m
  \item initbells.m
  \item playbells.m
  \item drawtime.m
  \item setup/APS\_DFD\_setup.dat (or a file name of your choice; specified in APS\_DFD.m)
  \item 1bell.wav
  \item 2bells.wav
  \item 3bells.wav
 \end{itemize}

 \item[2.]  Have the schedule of the conference ready and save it as ``setup/APS\_DFD\_setup.dat''.  Give the hours in 24-hour format, e.g. 12:00 AM is 0 hours 0 minutes and 12:00 PM is 12 hours 0 minutes.  For simplicity, you can edit this using Microsoft Excel.  APS\_DFD.m will read in the schedule.  The last line sets the current day (which is 1 right now).  If you have to restart due to a change in schedule or otherwise, you need to set this to the appropriate day.
 \item[3.]  Define your preferred font in the ``APS\_DFD.m'' file.
 \item[4.]  Define whether or not you want to play Bells ('Y' or 'N')
 \item[5.]  Define your platform ('lin', 'win', or 'mac') (because the best sound command is different on each platform)

 \item[6.]  Run APS\_DFD.m and you should see a screen come up showing the title of the conference and the first session.

 \item[7.]  Delays can be implemented by pressing the '$<$' and '$>$' keys on the keyboard while in the screen:
   \begin{itemize}
    \item '$>$' instructs the program to add 1 minute delay to the schedule,
    \item '$<$' instructs the program to subtract 1 minute delay to the schedule.
   \end{itemize}
   In other word, use '$>$' when the schedule is running late and '$<$' when the schedule is running fast.


\end{itemize}



\parskip -0.1in

\end{document} 
